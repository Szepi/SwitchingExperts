\documentclass{article} % For LaTeX2e
\usepackage[utf8]{inputenc}
\usepackage{nips15submit_e,times}
\usepackage{times}
%\usepackage{hyperref}
\usepackage{url}
%\documentstyle[nips14submit_09,times,art10]{article} % For LaTeX 2.09
\nipsfinalcopy

\title{Conservative Bandits}


\author{
Blah blah
}

% The \author macro works with any number of authors. There are two commands
% used to separate the names and addresses of multiple authors: \And and \AND.
%
% Using \And between authors leaves it to \LaTeX{} to determine where to break
% the lines. Using \AND forces a linebreak at that point. So, if \LaTeX{}
% puts 3 of 4 authors names on the first line, and the last on the second
% line, try using \AND instead of \And before the third author name.

% \newcommand{\fix}{\marginpar{FIX}}
% \newcommand{\new}{\marginpar{NEW}}

%\nipsfinalcopy % Uncomment for camera-ready version

%\usepackage{newtxtext}
\usepackage{textcomp}
\usepackage{amsmath,mathtools}
\usepackage{amsthm,thmtools,thm-restate}
\usepackage{amssymb} % loads amsfonts
%\usepackage[cmintegrals,bigdelims]{newtxmath}
\usepackage[scr=rsfso]{mathalfa}
\usepackage{dsfont}
\usepackage{bm}

%\usepackage{microtype}
\usepackage[inline,shortlabels]{enumitem}
\usepackage{setspace}
\usepackage{tikz-cd}
\usepackage{booktabs}
\usepackage[boxed]{algorithm}
\usepackage{algpseudocode}
%\usepackage{fullpage}
%\usepackage{dblfloatfix}

%%%%%%%%%%%%%%%%%%%%%%%%%%%%%%%%
% BIBLIOGRAPHY/CROSS-REFERENCES
%%%%%%%%%%%%%%%%%%%%%%%%%%%%%%%%

\usepackage[numbers]{natbib}

\usepackage{varioref}
%\usepackage[hidelinks]{hyperref}
\usepackage[colorlinks=true,citecolor=blue]{hyperref}
\usepackage[capitalise]{cleveref}

%%%%%%%%%%%%%%%%%%%%%%%%%%%%%%%%
% TO-DO NOTES
%%%%%%%%%%%%%%%%%%%%%%%%%%%%%%%%
\setlength{\marginparwidth}{10ex}%added by Yifan so that the todos can fit in the margin
\usepackage[obeyFinal]{todonotes}

\newcommand{\tinytodo}[2][]{\todo[size=\tiny]{#2}}
%\newcommand{\tinytodo}[2][]{\todo[size=\tiny, #1]{\begin{spacing}{1.0}#2\end{spacing}}}
\newcommand{\todoc}[2][]{\tinytodo[color=blue!50, #1]{Cs: #2}} % Csaba
\newcommand{\todod}[2][]{\tinytodo[color=red!20, #1]{D:\@#2}} % David
\newcommand{\todop}[2][]{\tinytodo[color=red!20, #1]{P:\@#2}} % Peter

%%%%%%%%%%%%%%%%%%%%%%%%%%%%%%%%
% ROSHAN's MACROS
%%%%%%%%%%%%%%%%%%%%%%%%%%%%%%%%

\newcommand{\wildcard}{\mathinner{\,{\cdot}\,}}
\newcommand{\defeq}{\coloneq}
\newcommand{\eqdef}{\eqcolon}
\newcommand{\Real}{\mathds{R}}
\newcommand{\Nat}{\mathds{N}}
\newcommand{\Int}{\mathds{Z}}
\newcommand{\Ex}{\mathds{E}}
\renewcommand{\Pr}{\mathds{P}}
\newcommand{\UCB}{\mathrm{UCB}}
\newcommand{\LCB}{\mathrm{LCB}}
\renewcommand\mid{\mathinner{\vert}}
\DeclareMathOperator*{\argmin}{arg\,min}
\DeclareMathOperator*{\argmax}{arg\,max}
\newcommand\given[1][]{\nonscript\:#1\vert\nonscript\:\mathopen{}\allowbreak}

\DeclarePairedDelimiterX\set[1]\{\}{%
  \renewcommand\given{\nonscript\:\delimsize\vert\nonscript\:\mathopen{}\allowbreak}
  #1
}

\newcommand{\one}[1]{\mathds{1}\set{#1}}

%%%%%%%%%%%%%%%%%%%%%%%%%%%%%%%%
% TOR's MACROS
%%%%%%%%%%%%%%%%%%%%%%%%%%%%%%%%

\newcommand{\bset}[1]{\left\{#1\right\}}
\newcommand{\eqn}[1]{\begin{align}#1\end{align}}
\newcommand{\eq}[1]{\begin{align*}#1\end{align*}}
\newcommand{\ind}[1]{\mathds{1}\!\!\bset{#1}}
\newcommand{\ceil}[1]{\left \lceil {#1} \right\rceil}
\renewcommand{\P}[1]{\mathbb{P}\left\{#1\right\}}
\newcommand{\floor}[1]{\left \lfloor {#1} \right\rfloor}
\newcommand{\ubar}[1]{\underline{#1\mkern-3mu}\mkern3mu }

%%%%%%%%%%%%%%%%%%%%%%%%%%%%%%%%
% YIFAN's MACROS
%%%%%%%%%%%%%%%%%%%%%%%%%%%%%%%%

\newcommand{\iset}[1]{\left[#1 \right]}
\newcommand{\cset}[2]{\left\{#1\,:\,#2\right\}}

\newcommand{\rar}{\rightarrow}
\newcommand{\Rar}{\Rightarrow}
\newcommand{\inprod}[2]{\left\langle #1 , #2 \right\rangle}
\newcommand{\norm}[1]{\left\Vert #1 \right\Vert}
\newcommand{\Prb}[1]{\Pr\left( #1 \right)}
\newcommand{\Prob}[1]{\Pr\left( #1 \right)}
\newcommand{\E}[1]{\mathbb{E}\left[ #1 \right]}
\newcommand{\EE}[1]{\mathbb{E}\left[ #1 \right]}
\newcommand{\EEt}[1]{\mathbb{E}_t\left[ #1 \right]}
\newcommand{\R}{\mathbb{R}}
\newcommand{\muhat}{\hat{\mu}}
\newcommand{\eps}{\varepsilon}

\allowdisplaybreaks

%%%%%%%%%%%%%%%%%%%%%%%%%%%%%%%%
% THEOREMS
%%%%%%%%%%%%%%%%%%%%%%%%%%%%%%%%

\theoremstyle{plain}
\newtheorem{theorem}{Theorem}
\newtheorem{proposition}[theorem]{Proposition}
\newtheorem{lemma}[theorem]{Lemma}
\newtheorem{corollary}[theorem]{Corollary}
\theoremstyle{definition}
\newtheorem{definition}[theorem]{Definition}
\newtheorem{assumption}[theorem]{Assumption}
\newtheorem{remark}[theorem]{Remark}
\newtheorem{example}[theorem]{Example}
\theoremstyle{remark}


\begin{document}


\maketitle

\begin{abstract}
Blah 
\end{abstract}


\section{Adaptive Adversaries force Good Learners to Switch Often}
We construct an adaptive adversary that forces any low regret learner to switch linearly often.

Let $K$ be the number of experts and fix a learner.
Let $T>0$ be the number of rounds, $I_1,\dots, I_T$ be the choices of some fixed algorithm.
We will construct the losses $\ell_1,\dots,\ell_T: [K] \to [0,1]$ recursively.
First, let $\ell_1(i) = 0$ for all $i\in [K]$.
Assume that the losses are constructed up to time $t-1\ge 1$. Then, $\ell_t$ is constructed as follows.
Let
\begin{align*}
p_t(i) = \Prob{I_t = i|I_1,\ell_1, \dots, I_{t-1},\ell_{t-1} }\,
\end{align*}
and $p_t^* = \max_i p_t(i)$.
Define
\begin{align*}
\ell_t(i) = 
\begin{cases}
1, & \text{if } p_t(i) = p_t^*\,;\\
0, & \text{otherwise}\,,
\end{cases}
\end{align*}
where ties are broken randomly (i.e., the loss of one is assigned to a single action and the loss is assigned to some of the candidates at random, if there were multiple experts $i$ with $p_t(i) = p_t^*$).

The first observation is that $T = \sum_i \sum_t \ell_t(i) \ge K\min_i \sum_t \ell_t(i)$, hence 
\begin{align}
\label{eq:smalltotalloss}
\min_{i\in [K]} \sum_{t=1}^T \ell_t(i) \le T/K\,.
\end{align}
Now, assume that on an event  $\mathcal{E}_0$ (that holds with high probability) the regret of the learner is bounded by $C \sqrt{T}$:
\begin{align*}
{ \sum_{t=1}^T \ell_t(I_t)  - \min_{i\in [K]} \sum_{t=1}^T \ell_t(i) } \le C \sqrt{T}.
\end{align*}
Let $\EEt{X} = \EE{ X | I_1,\ell_1,\dots,I_{t-1},\ell_{t-1}}$.
Since $\sum_{t=1}^T \ell_t(I_t)$ and $\sum_{t=1}^T \EEt{\ell_t(I_t)}$ are within $C\sqrt{T}$ of each other with high probability
(because their difference is a martingale with bounded increments),
there is an event $\mathcal{E}$ that also holds with high probability where
\begin{align}
\label{eq:regretbound}
{ \sum_{t=1}^T \ell_t(I_t)  - \min_{i\in [K]} \sum_{t=1}^T \ell_t(i) } \le C \sqrt{T}
\end{align}
holds ($C$ is a universal constant that changes between the lines -- sorry).
In what follows all arguments hold on this event.

We have
\begin{align*}
 \sum_{t=1}^T \EEt{\ell_t(I_t) } =  \sum_{t=1}^T \sum_i \ell_t(i) p_t(i)  = \sum_{t=1}^T p_t^* \,.
\end{align*}
Combining this with~\eqref{eq:regretbound} and \eqref{eq:smalltotalloss}, we get
\begin{align*}
\sum_{t=1}^T \left\{{p_t^*} - \frac{1}{K}\right\}  \le C \sqrt{T}\,.
\end{align*}
This implies that for any $\epsilon>0$, for $t=1,\dots,T$, 
the condition ${p_t^*} - \frac{1}{K}>\eps$ holds at most $C \sqrt{T}/\eps$ times.
Choose $\eps = 2/(C \sqrt{T})$ to conclude that
${p_t^*} > \frac{1}{K} + 2/(C \sqrt{T})$ happens at most $T/2$ times, or, equivalently,
\begin{align}
{p_t^*} \le \frac{1}{K} + \frac{2}{C \sqrt{T}}
\label{eq:ptsmall}
\end{align}
happens at least $T/2$ times.
Fix an index $t$ when~\eqref{eq:ptsmall} holds.
Now, since for any $i$,
$1 = \sum_j {p_t(j)} \le {p_t(i)} + (K-1) {p_t^*} \le {p_t(i)} + (K-1) \{\frac{1}{K} + 2/(C \sqrt{T})\}$.
Reordering this, since the inequality holds for any $i$, we get
\begin{align*}
\min_i {p_t(i)} \ge \frac{1}{K} - \frac{2(K-1)}{C \sqrt{T}}\,.
\end{align*}
At these time point, the probability of switching is bounded away from zero (assuming $T$ is large compared to $K^2$).
Since this holds for at least $T/2$ indices, on the said event, the expected number of switches is linear.

\renewcommand{\bibsection}{\subsubsection*{\refname}}

\bibliographystyle{unsrtnat}
%\bibliography{../references} % chktex 11



\end{document}

%%% Local Variables:
%%% mode: latex
%%% TeX-master: t
%%% End:
